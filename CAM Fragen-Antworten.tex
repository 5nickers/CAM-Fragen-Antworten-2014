\input{header.tex}
\usepackage{paralist}
\begin{document}

\maketitle

Dieser Text ist unter der
\href{http://creativecommons.org/licenses/by-nc/4.0/}{Creative Commons CC BY-NC 4.0}
Lizenz veröffentlicht.

\textcolor{red}{%
    Ich erhebe keinen Anspruch auf Vollständigkeit oder Richtigkeit. Falls ihr
    Fehler findet oder etwas fehlt, dann meldet euch bitte über den
    Emailkontakt.
}

\tableofcontents

\newpage

\section{Einführung CAM}

\subsection*{%
    Beschreiben des Produktlebenszyklus: Phasen. Wie verhalten sich Umsatz und
    Gewinn in Abhängigkeit von der jeweiligen Phase. Begründen Sie Ihre
    Aussage.
}


\begin{figure}[h]
    \centering
    \includegraphics[scale=0.7]{Bild1_1.png}
\end{figure}

In der Graphik können wir die einzelnen Phasen ablesen, in der Entwicklung
kostet das Produkt Geld, da wir Leute, Maschinen etc für die Entwicklung
bezahlen müssen. Auch in der Einführungsphase beginnt der Umsatz zu steigen,
allerdings kostet uns das Produkt noch Geld, da wir Geld in Werbung und oftmals
Ausbesserungen stecken müssen. In der Wachstumsphase fangen wir an mit dem
Produkt Geld zu verdienen, was in der Phase der Produktreife zu einem Umsatz
als auch Gewinnmaximum führt. In den Phasen der Sättigung und des Rückgangs
sinken der Umsatz als auch der Gewinn wieder. Zum Schluss kostet uns das
Produkt wieder Geld, da wir trotz wenig bis keinen Verkäufen noch Support oder
Garantieleistungen erfüllen müssen.

\subsection*{%
    Welche Potenziale kann der Anwender von der Kopplung CAD/CAM erwarten?
    (fünf Beispiele)
}

\begin{enumerate}[1)]
    \item
        die direkte Anbindung an die Fertigung ermöglicht einen direkteren
        Austausch zwischen den Konstrukteuren und den Leuten an der Maschine
    \item
        Kostensenkung: durch weniger Fehlkonstruktionen
    \item
        Zeitersparnis: es kann schon mit CAM angefangen werden bevor CAD
        komplett fertig ist
    \item
        keine Datenumwandlungen: bei verschiedenen Systemen bei CAD und CAM
        kann es sein, das man Daten umwandeln muss, dabei gehen oft Parameter
        und/oder Daten verloren
    \item
        schnelle Ermittlung von Problemen
\end{enumerate}

\newpage

\subsection*{%
    Was bedeuten die Kürzel und welche Tätigkeiten werden mit diesen Systemen
    unterstützt?  Produkt entwerfen $\rightarrow$ CAD $\rightarrow$ CAE
    $\rightarrow$ CAP $\rightarrow$ CAM $\rightarrow$ Produkt
}

\begin{description}
    \item[CAD]
        Computer Aided Design $\Rightarrow$ Konstruieren am Rechner
    \item[CAE]
        Computer Aided Engineering $\Rightarrow$ umfasst alle Varianten der
        Rechner-Unterstützung von Arbeitsprozessen in der Technik.
    \item[CAP]
        Computer Aided Process Planning $\Rightarrow$ Werkzeugmaschinen
        belegen, Fertigungsmittel bereitstellen
    \item[CAM]
        Computer Aided Manufacturing $\Rightarrow$ Spannvorrichtung festlege,
        Werkzeuge auswählen, Arbeitspläne erstellen, NC-Progs simulieren und
        bereitstellen, Produkt herstellen
\end{description}

\subsection*{%
    Welche Informationen / Unterlagen erhalten Sie aus den Bereichen CAD, CAE,
    CAP, CAM?
}

\begin{description}
    \item[CAD] Zeichnungen, Stücklisten
    \item[CAE] Montagepläne, Doku
    \item[CAP] Maschinenbelegung, Arbeitspläne
    \item[CAM] NC-Programme, Werkzeuglisten, Bestückungspläne, Korrekturen
\end{description}

\newpage

\section{NC-Programmierung}

\subsection*{Was bedeuten die Kürzel: CNC, DNC, LAN?}

\begin{description}
    \item[CNC] Computerized Numerical Control
    \item[DNC] Direct Numerical Control
    \item[LAN] Local Area Network
\end{description}

\subsection*{%
    Welche Informationen / Unterlagen liefert die Arbeitsplanung im Rahmen der
    technischen Auftragsbearbeitung?
}

\begin{itemize}
    \item Arbeitspläne
    \item NC-Programme
    \item Einrichteblätter
    \item Montageblätter
\end{itemize}

\subsection*{
    Welche Informationen enthält ein Arbeitsplan? Erstellen Sie ein Beispiel
    für einen Arbeitsplan.
}

Blatt, Datum, Bearbeiter, Auftragsnummer, Stückzahl, Bereich, Benennung,
Zeichnungsnummer, Werkstoff, Rohform und -abmessungen, Rohgewicht,
Fertigungsgew.

tabellarisch: AVG-Nr, Arbeitsvorgangsbeschreibung, Kostenstelle, Lohngruppe,
Masch.gruppe, Fertigungshilfsmittel

\begin{figure}[h]
    \centering
    \includegraphics[scale=0.7]{Bild2_3.png}
\end{figure}

\subsection*{%
    Welche Informationen enthält ein Einrichteblatt und welche Abteilung
    arbeitet mit diesen Informationen?
}

Werkzeugliste, Aufspannplan, NC-Arbeitsplan, Messmittel

Abteilung:??

\subsection*{%
    Nennen Sie drei verschiedene Möglichkeiten zur Programmierung von
    NC-Maschinen und beschreiben Sie in Stichworten die Unterschiede
}

\begin{enumerate}[1)]
    \item direkte Programmierung an der Maschine
    \item direkte Programmierung an einem externen Gerät
    \item Programmierung mit Hlfe einer graphischen Oberfläche und interaktiv
    \item Programmierung mit Hilfe einer grafschen Oberfäche und CAD-Kopplung
    \item
        Programmierung mit Hilfe einer grafschen Oberfäche und integriertem
        CAM-System
\end{enumerate}

\subsection*{%
    Was versteht man unter „grafsch interaktiver NC-Programmierung“. Erläutern
    Sie drei Merkmale in Stichworten
}

\begin{itemize}
    \item Bearbeitungsgeometrie des Werkstücks graphisch auf dem Bildschirm
    \item
        interaktive Eingabevorgänge $\Rightarrow$ im Dialog eingegeben,
        kontrolliert und korrigiert werden
    \item Schritt für Schritt Programmierung
    \item Möglichkeit zur Kopplung an CAD
\end{itemize}

\subsection*{%
    Nennen Sie beispielhaft 5 Einsatzkriterien für grafsch interaktive
    Programmierung
}

\begin{itemize}
    \item komplizierte Teile 
    \item mehr als zwei Achsen 
    \item Anzahl der Maschinen 
    \item Anzahl der Bearbeitungen 
    \item Typenvielfalt der Maschinen 
    \item Typenvielfalt der Steuerung 
    \item Automatisierungsgrad der NC-Maschinen 
    \item komplizierte Technologie
    \item vereinheitlichte Technologie 
    \item geringe Wiederholhäufigkeit 
    \item CAD-Kopplung
\end{itemize}

\subsection*{%
    Nennen Sie 3 Bezeichnungen von Schnittstellen für den Datenaustausch von
    CAD zu CAM-Systemen
}

\begin{description}
    \item[VDAFS]
        Verband der Automobilindustrie – Flächenschnittstelle $\Rightarrow$
        Austausch von 3D-Freiform-Kurven und Flächendaten 
    \item[STEP] Standard for the Exchange of Product Model Data 
    \item[IGES] Initial Graphics Exchange Specification 
    \item[DXF] Drawinf Exchange Format 
    \item[STL] Standard Triangulation Language 
\end{description}

\newpage

\section{Features und Werkzeugorga}

\subsection*{%
    Die Feature Technologie besitzt ein erhebliches Potenzial zu Beschleunigung
    der Produktentwicklung. Welches Problem soll mit Hilfe der
    \emph{Feature-Technologie} behoben werden?
}

Es wird nicht mehr nur die geometrischen Elemente mit ihren festen Werten
erfast, sondern auch die hinterlegten Informationen die sogenannte „Semantik“
wie z.\.B. Funktionen und Fertigungstechnologie.

Ein Feature ist also eine spezifische aus Daten generierte Sichtweise auf die
Produktbeschreibung, die Eigenschaftsklassen und bestimmte Phasen des
Produktlebenszyklus in sich bündelt.

\subsection*{%
    Beschreiben sie in Stichpunkten den Unterschied zwischen \glqq
    konventionellen\grqq {} und feature basierten CAD-Systemen.
}

siehe oben.

\subsection*{%
    Erläutern sie an einem Beispiel den Unterschied zwischen \glqq
    Konstriktionsfeature\grqq {} und \glqq Fertigungsfeature\grqq
}

Allgemein kann man sagen, dass das Konstruktionsfeature Informationen über
Teile, Material und Funktion enthält. Das Fertigungsfeature enthält
Informationen wie die Bearbeitungsschritte (bohren, senken, …)

\begin{center}
    \begin{figure}[h]
        \begin{minipage}[hbt]{7cm}
            \includegraphics[width=6cm]{Bild3_3_1.png}
        \end{minipage}
        %
        \begin{minipage}[hbt]{7cm}
            \includegraphics[width=6cm]{Bild3_3_2.png}
        \end{minipage}
    \end{figure}
\end{center}

\subsection*{Welche Fragen gibt es zur Featureerkennung?}

\begin{itemize}
    \item Welche Maschine?
    \item Welches Material?
    \item Welches Werkzeug?
    \item Welche Aufspannung?
    \item Welche Technologie?
    \item Welche Bearbeitungsreihenfolge?
\end{itemize}

\subsection*{%
    Welche Bedeutung besitzt die Werkzeugorga im Rahmen von CAM. Erläutern sie
    in Stichworten.
}

\begin{itemize}
    \item Bessere Verfügbarkeit der Werkzeuge
    \item Reduzierung der Werkzeugtypen
    \item Reduzierung der Werkzeuganzahl
    \item Komponenteneinsparung durh Technologiestandardisierung
    \item Bessere Nutzung der Ressourcen (Wiederverwendung, Reststandzeit)
    \item Vereinfachte Arbeitsplanung
    \item Vereinfachte, vereinheitlichte NC-Programmierung mit Datensicherheit
\end{itemize}

\subsection*{%
    Erläutern sie beispielhaft, warum im Zusammenhang mit dem Werkzeug eine
    Kollisionsbetrachtung notwendig sein kann.
}

Angenommen man hat recht weit unten in einem Werkstück eine Nut, die seitlich
in die Wand gefräst werden soll. Eine Kollisionsbetrachtung ist hier notwendig,
da man nicht genau weiß, wie der Werkzeugkopf aufgebaut ist und wie die
Maschine ihn versuchen wird in die notwendige Position zu bringen. Es könnte ja
passieren, das der Operationsraum zu klein ist.

\end{document}
